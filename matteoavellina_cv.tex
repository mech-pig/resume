\documentclass[a4paper,MMMyyyy,nonstopmode]{simpleresumecv}

\setmainfont[JetBrainsMono Nerd Font]{JetBrainsMono Nerd Font}

\renewcommand{\BulletSymbol}{{\normalfont\fontsize{6.5}{7.8}\selectfont\raisebox{0.17em}{\char"2022}}}
\renewcommand{\SubBulletSymbol}{{\normalfont\fontsize{6}{7.2}\selectfont\raisebox{0.17em}{\char"2022}}}
\renewcommand{\SubSubBulletSymbol}{{\normalfont\fontsize{6}{7.2}\selectfont\raisebox{0.17em}{\char"2022}}}

\newcommand{\CVAuthor}{Matteo Avellina}
\newcommand{\CVNote}{In compliace with the GDPR and the Italian Legislative Decree no. 196 dated 30/06/2003, I hereby authorize you to use and process my personal details contained in this document. CV compiled on {\today}}
\newcommand{\CVWebpage}{https://mechpig.me}
\newcommand{\CVGithub}{https://github.com/mech-pig}
\newcommand{\CVLinkedin}{https://linkedin.com/in/matteo-avellina}
\newcommand{\CVEmail}{matteo.avellina@gmail.com}
\newcommand{\CVMobile}{+39\,328\,7685221}

\newcommand{\DoubleBigGap}{\BigGap\BigGap}


% PDF settings and properties.
\hypersetup{
  pdftitle={\CVTitle},
  pdfauthor={\CVAuthor},
  pdfsubject={\CVWebpage},
  pdfcreator={XeLaTeX},
  pdfproducer={},
  pdfkeywords={},
  unicode=true,
  bookmarks=true,
  bookmarksopen=true,
  pdfstartview=FitH,
  pdfpagelayout=OneColumn,
  pdfpagemode=UseOutlines,
  hidelinks,
  breaklinks,
}

%%%%%%%%%%%%%%%%%%%%%%%%%%%%%%%%%%%%%%%%%%%%%%%%%%%%%%%%%%%%%%%%%
%% ACTUAL DOCUMENT.
%%%%%%%%%%%%%%%%%%%%%%%%%%%%%%%%%%%%%%%%%%%%%%%%%%%%%%%%%%%%%%%%%

\begin{document}

%%%%%%%%%%%%%%%
% TITLE BLOCK %
%%%%%%%%%%%%%%%

  \Title{\CVAuthor}

  \begin{SubTitle}
    \href{https://www.google.com/maps/place/Via+dei+Fiordalisi,+1,+20146+Milano+MI}{Via Dei Fiordalisi, 1 - 20146 Milano, MI, Italy}
    \Gap
    \href{mailto:\CVEmail}{\CVEmail}\enspace\BulletSymbol\enspace\href{\CVWebpage}{\CVWebpage}\enspace\BulletSymbol\enspace\CVMobile
    \Gap
    \href{\CVLinkedin}{\CVLinkedin}\enspace\BulletSymbol\enspace\href{\CVGithub}{\CVGithub}
  \end{SubTitle}

  \begin{Body}

    \Section
    {Whoami}
    {Whoami}
    {PDF:Whoami}
    \begin{Detail}
      I'm a software developer living in Milan, Italy.
      \par
      A large part of my job consists of designing and building web APIs and helping others to do the same. I work mainly in Python and Node.js, but I have an interest in programming languages and software development at large. I always try to write clean, modular and concise code.
      \par
      I value pragmatic minimalism, clean architecture, microservices, automation and immutability, and try to pursue them whenever possible.
    \end{Detail}

    \Section
    {Work Experience}
    {Work Experience}
    {PDF:Work Experience}

      \Entry
        \href{https://www.glickon.com}{\textbf{Glickon}, Milano, MI, Italy}
        \Gap
        \textit{Senior Backend Developer}
          \hfill
          \DatestampYM{2020}{06} -- Present
        \Gap
        \begin{Detail}
          \Gap
          \BulletItem
            designed and implemented python-based web apis and services (Django, fastapi), backed by PostgreSQL and MongoDB
          \BulletItem
            setup continuous integration pipelines with Github Actions (static analysis, unit and integration testing, deploying to staging environment)
          \BulletItem
            advocated the use of containers for software development and deployment, contributed to the migration of all production services from Heroku to Google Cloud Platform (Cloud Run)
          \BulletItem
            developed multi-tenant single sign-on integrations with customers' identity providers, based on open source federation protocols (SAML, oAuth2/OIDC) 
        \end{Detail}
      
      \DoubleBigGap

      \Entry
        \href{https://igenius.ai}{\textbf{iGenius}, Milano, MI, Italy}
        \Gap
        \textit{Team Leader Backend}
          \hfill
          \DatestampYM{2019}{02} -- \DatestampYM{2020}{05}
        \Gap
        \begin{Detail}
          \Gap
          \BulletItem
            lead a team of 10+ people, supporting them in their daily duties with mentoring, 1:1s, reviews and retrospectives
          \BulletItem
            contributed to the design of the backend architecture, using a service-oriented architecture approach built on top of Docker and Kubernetes
          \BulletItem
            contributed to the product development from high-level requirements to code, following the best practices in software development (clean code, clean architecture, DDD, event sourcing) and the principles of the Agile manifesto
          \BulletItem
            brought 8 new team members on board, conducted more than 35 technical interviews, screened more than 120 resumes
        \end{Detail}

        \DoubleBigGap

        \textit{Backend Developer}
          \hfill
          \DatestampYM{2017}{03} -- \DatestampYM{2019}{02}
        \begin{Detail}
          \Gap
          \BulletItem
            designed and implemented backend services and web apis in python (falcon, flask, fastapi) and node.js (express)
          \BulletItem
            contributed to the transition from a monolithical architecture to a service oriented one, based on \href{https://www.docker.com/}{Docker} and deployed on \href{https://aws.amazon.com/}{AWS}
          \BulletItem
            worked with \href{https://redis.io/}{redis} and \href{https://www.postgresql.org/}{PostgreSQL}, managing schema migrations with \href{https://sqitch.org/}{sqitch}
        \end{Detail}

      \DoubleBigGap

      \Entry
        \href{http://mammaitaliafood.com}{\textbf{MammaItalia}, Milano, MI, Italy}
        \Gap
        \textit{Backend Developer}
          \hfill
          \DatestampYM{2016}{01} -- \DatestampYM{2016}{06}
        \Gap
        \begin{Detail}
          \Gap
          \BulletItem
            developed a rest api service that lists grocery stores selling a given product based on user location
          \BulletItem
            worked with \href{https://www.djangoproject.com/}{Django} with \href{https://www.django-rest-framework.org/}{Django REST Framework}, backed by \href{https://www.postgresql.org/}{PostgreSQL} with \href{https://postgis.net/}{PostGIS} extension
        \end{Detail}

    \Section
    {Education}
    {Education}
    {PDF:Education}

      \Entry
        \href{http://www.polo-como.polimi.it}{\textbf{Politecnico di Milano - Polo Territoriale di Como}, Como, CO, Italy}
        \Gap
        \textit{M.Sc., Computer Science and Engineering - Sound and Music Engineering}
          \hfill
          \DatestampY{2012} -- \DatestampY{2016}
        \Gap
        110/110 \textit{cum laude}

      \DoubleBigGap

      \Entry
        \href{https://www.polimi.it}{\textbf{Politecnico di Milano}, Milano, MI, Italy}
        \Gap
        \textit{B.Sc., Computer Science and Engineering}
          \hfill
          \DatestampY{2009} -- \DatestampY{2012}
        \Gap
        102/110

      \DoubleBigGap

      \Entry
        \href{https://www.eliovittorini.edu.it/}{\textbf{Liceo Scientifico Elio Vittorini}, Milano, MI, Italy}
        \Gap
        \textit{Diploma di Maturità Scientifica}
          \hfill
          \DatestampY{2004} -- \DatestampY{2009}
        \Gap
        100/100

    \Section
    {Publications}
    {Publications}
    {PDF:Publications}

    \href{https://doi.org/10.1002/acs.2803}
    {\underline{Avellina, M.}, Brankovic, A., and Piroddi, L.,
    ``Distributed randomized model structure selection for NARX models.'',
    \textit{International Journal of Adaptive Control and Signal Processing},
    vol.~31
    (2017)
    doi:10.1002/acs.2803}

    \Section
    {Skills}
    {Skills}
    {PDF:Skills}
      \Entry
        \textbf{Programming Languages}
        \Gap
        \BulletItem
          Python
        \BulletItem
          Node.js
        \BulletItem
          Haskell (Basic)

      \BigGap

      \Entry
        \textbf{Software Development}
        \Gap
        \BulletItem
          Rest API Design
        \BulletItem
          Clean Code
        \BulletItem
          Test-Driven Development
        \BulletItem
          Clean Architecture
        \BulletItem
          Microservices
        \BulletItem
          Domain-Driven Design
        \BulletItem
          Functional Programming

      \BigGap

      \Entry
        \textbf{Integration \& Deployment}
        \Gap
        \BulletItem
          GitlabCI, Github Actions
        \BulletItem
          Docker
        \BulletItem
          Kubernetes (Basic)

      \BigGap

      \Entry
        \textbf{Databases}
        \Gap
        \BulletItem
          PostgreSQL
        \BulletItem
          redis
        \BulletItem
          MongoDB


    \Section
    {Languages}
    {Languages}
    {PDF:Languages}

      \Entry
        Italian: Native language.
      \Gap
      \Entry
        English: Proficient (speaking, reading, writing).
      \Gap
      \Entry
        German: Basic.

  \end{Body}

  \DoubleBigGap
  \UseNoteFont
  \null
  \textit{\CVNote}


\end{document}
